\chapter*{謝辞}%
\addcontentsline{toc}{chapter}{謝辞}

% 一般的に論文における「謝辞」(acknowledgments)とは、その論文を作成する上で不可欠だった様々な支援に対する感謝の気持ちを述べる場所です。通常の投稿論文であれば、その論文の作成者・共同研究者は主著者(筆頭著者もしくは責任著者)や共著者として著者リストに入っています。しかし修士論文は単著で書くため、例えば指導教員から論文のアイデアをもらっても指導教員は共著者になりません。また様々な共同研究者にデータ解析を手伝ってもらったり、実験データを提供してもらった場合にも、これが普通の論文であれば共著者になりうるところですが、修士論文では単著、つまりあなたの名前だけが記載されます。そこで、謝辞の必要性が生じるのです\footnote{複数著者による投稿論文の場合、共著者にするべきか謝辞での言及のみに留めるかは、場合によります。}。

% 先輩の修論の謝辞を真似て、やたらと大人数への謝辞を並べてある修論をよく見かけます。私の所属する研究室は大所帯のため、教員全員、院生全員、事務補佐員が20人以上並んでいるものがあります。もちろん感謝を述べたかったら書けば良いですしそれを止めるつもりはありませんが、修論の読者からすると「別に大して感謝の気持ちのないくせに、取捨選択する度胸がなく、差し障りのないように機械的に羅列しただけだろう」という印象を持ち、逆に謝辞としての意味が薄れます。

% 絶対に謝辞に含めなくてはいけないのは、およそ次の通りです。順序は前後しても構いませんが、貢献度の高いものを前にするべきです。
% \begin{enumerate}
% \item 指導教員(教授、准教授など)
% \item 修士論文の研究テーマやアイデアを考えた人、提供した人(指導教員の場合が多い)
% \item 指導教員以外で直接的に指導した人(助教、ポスドクなど)
% \item 論文において本質的となる議論や指摘を行ってくれた人
% \item 実験やデータ解析をかなり手伝ってくれた人(先輩や共同研究者など)
% \item データや解析スクリプトなどを提供してくれた人(共同研究者など)
% \item 研究資金を受給していれば、その資金名と提供元(学内の研究支援も含む)
% \end{enumerate}

% 以上は投稿論文にも当てはまります。修士論文の場合、さらに同期や秘書さんなどを謝辞に加える学生が多いです。これは科学論文として必須ではありませんので、謝辞に加えないからといって失礼に当たるわけではありません。また家族への謝辞を加えることもよくあります\footnote{独立生計でない場合、研究資金の提供先ですので当然と言えば当然かもしれません。}。

% また謝辞では次のことに注意してください。

% \begin{enumerate}
% \item 現在は指導教「員」と呼称し、指導教「官」ではない\footnote{国立大学法人化によって、大学教員は公務員ではなくなったため。}。
% \item 職階を間違えないこと。「准教授」を「助教授」、「助教」を「助手」とする間違いが多い\footnote{2007年に名称が変更となった。}。
% \item 氏名の漢字を絶対に間違えない。旧字体(曉と暁など)や異体字(齋藤と斎藤など)に気をつけること。
% \item 所属先の正式名称を書くこと。
% \item 敬称をつけること(「2年間にわたり御指導くださった XX 教授」など)。博士号を持っている人には「博士」をつけること。
% \end{enumerate}

本研究を遂行するにあたって、多くの方々にお世話になりました。まずは指導教員である奥村恭幸准教授には大変お世話になりました。研究の構想から、詳細な技術各論に至るまでたくさんの有用なアドバイスをいただき、本研究をスムーズに進めることができました。また、CERN出張やシンポジウムなど多くの学びの機会を提供していただき、様々な面での成長をサポートしていただきました。石野雅也教授にも大変お世話になりました。研究に必要な環境の構築や、研究内容に対する的確な助言をいただき、研究を深めることができました。博士課程進学後もよろしくお願いします。

日々の研究に関して多くの助言をくださった、研究室ミーティングに参加してくださっているスタッフの皆様にも心から感謝申し上げます。斎藤智之助教、増渕達也助教には現行システムの観点から様々な助言を頂き大変参考になりました。他のICEPPの皆様にもお世話になりました。田中純一教授、澤田龍准教授にはセミナーなどで質問やコメントをいただき、研究をさらに深めることができました。ICEPP夏の学校やワークショップでは、他にも多くのICEPPのスタッフや先輩方にお世話になり、充実した研究生活を送ることができました。また秘書の皆様にも、出張手続きをはじめとした様々な面で支えていただきました、感謝申し上げます。

共同で高輝度LHC-ATLAS実験に向けたアップグレード研究を行なっている皆様にも、大変お世話になりました。KEKの佐々木修氏には、ASDやPS boardの取り扱いについて助言をいただき、大変助かりました。堀井泰之氏にはPS boardやSLの開発についてたくさん相談させていただきました。また、名古屋大学への出張の際にもサポートしていただきお世話になりました。戸本誠氏や前田順平氏には、TGC会合やシンポジウムなどで研究に対するコメントを頂き、研究を進めるにあたり大変参考になりました。青木雅人氏には、TGC検出器のエキスパートとしてCERNでの活動を牽引してくださり、専門的な知識を多く学ぶことができました。河本さん、辻川さん、三森さん、大隅さん、鍋山さん、橋爪さん、中川くんにはSLやPS boardの開発に際して、様々なアドバイスをいただき感謝しています。皆様との議論や試験を通して、共同研究としてグループで開発を進めることの面白さを実感することができました。今後ともよろしくお願いします。

研究室の先輩方にも大変お世話になりました。杉崎さん、田中さん、青木さん、林さん、三島さん、山下さんにはエレクトロニクスやTGCシステムについてたくさん教えていただきました。研究で詰まった際などは相談に乗っていただき、大変心強かったです。同期の長坂とは、日頃からたくさんの議論を共にさせてもらい、良い刺激をもらっています。これからも切磋琢磨していけたらと思います。名古屋大の橋本くん、和田さんとはPS board のQAQC試験に関して長い間研究を共に進めることができ、活発な意見交換を通じて研究を深めることができました。ありがとうございました。

以上に述べた他にも、多くの方々にお世話になりました。本研究の遂行に協力していただいたこと、深く感謝申し上げます。そして最後に、研究生活を支えてくれた家族に感謝します。

