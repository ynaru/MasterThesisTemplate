% jsbookで余白が広すぎるのを直す
% 参照 https://oku.edu.mie-u.ac.jp/~okumura/jsclasses/
\setlength{\textwidth}{\fullwidth}
\setlength{\evensidemargin}{\oddsidemargin}
\addtolength{\textwidth}{-5truemm}
\addtolength{\oddsidemargin}{5truemm}

% 同梱の ISEE 用の表紙テンプレ
\usepackage{thesis_cover}

% OTF フォントを使えるようにし、複数のウェイトも使用可能にする。
% これがないと、Mac のヒラギノ環境で使われる角ゴが太すぎてみっともない。
\usepackage[deluxe]{otf}

% OT1→T1に変更し、ウムラウトなどを PDF 出力で合成文字ではなくす
\usepackage[T1]{fontenc}

% uplatex の場合に必要な処理 
\usepackage[utf8]{inputenc} % エンコーディングが UTF8 であることを明示する。
\usepackage[prefernoncjk]{pxcjkcat} % アクセントつきラテン文字を欧文扱いにする

% Helvetica と Times を sf と rm のそれぞれで使う。
% default だとバランスが悪いので、日本語に合わせて文字の大きさを調整する。
\usepackage[scaled=1.05,helvratio=0.95]{newtxtext}

% 色
\usepackage[dvipdfmx]{color}

%naru added
\usepackage{enumitem}
\usepackage{ascmac}
\setcounter{tocdepth}{2}
\usepackage{physics}
\usepackage{multirow}
\usepackage{array}


% latexdiff
% 実際の修論には入れる必要なし
%DIF PREAMBLE EXTENSION ADDED BY LATEXDIFF
%DIF UNDERLINE PREAMBLE %DIF PREAMBLE
\RequirePackage[normalem]{ulem} %DIF PREAMBLE
\RequirePackage{color}\definecolor{RED}{rgb}{1,0,0}\definecolor{BLUE}{rgb}{0,0,1} %DIF PREAMBLE
\providecommand{\MyDIFadd}[1]{{\protect\color{blue}\uwave{#1}}} %DIF PREAMBLE
\providecommand{\MyDIFdel}[1]{{\protect\color{red}\sout{#1}}}                      %DIF PREAMBLE
%DIF SAFE PREAMBLE %DIF PREAMBLE
\providecommand{\MyDIFaddbegin}{} %DIF PREAMBLE
\providecommand{\MyDIFaddend}{} %DIF PREAMBLE
\providecommand{\MyDIFdelbegin}{} %DIF PREAMBLE
\providecommand{\MyDIFdelend}{} %DIF PREAMBLE
%DIF FLOATSAFE PREAMBLE %DIF PREAMBLE
\providecommand{\MyDIFaddFL}[1]{\MyDIFadd{#1}} %DIF PREAMBLE
\providecommand{\MyDIFdelFL}[1]{\MyDIFdel{#1}} %DIF PREAMBLE
\providecommand{\MyDIFaddbeginFL}{} %DIF PREAMBLE
\providecommand{\MyDIFaddendFL}{} %DIF PREAMBLE
\providecommand{\MyDIFdelbeginFL}{} %DIF PREAMBLE
\providecommand{\MyDIFdelendFL}{} %DIF PREAMBLE
%DIF END PREAMBLE EXTENSION ADDED BY LATEXDIFF
