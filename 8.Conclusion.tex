\chapter{結論と今後の展望}
\label{chap_conclusion}

2029年に開始される高輝度LHC-ATLAS実験では、初段トリガーレートは現行の100 kHzから1 MHzに、初段トリガーレイテンシーは2.5 $\mu\mathrm{s}$から 10 $\mu\mathrm{s}$に拡張される。これに伴い、TGC検出器システムも一部を除くすべてのエレクトロニクスが刷新され、PS boardからすべてのヒット信号をヒットの有無にかかわらずSLへと送信するようになる。これによりRun 3で複数のボードで分割されていたトリガーロジックは、SLのFPGA上に集約され、TGC BW 全7層のヒット信号を用いた柔軟かつ包括的なトリガーロジックが実現可能となる。

本研究ではSLのトリガー論理回路の統合を完了させ、その性能評価のためのシングルボード試験システムを開発した。まず、Wire Strip Coincidence、Inner Coincidence、Track Selectorの各ロジックの統合を進めた。クロックの分配やトリガーモジュール間の配線に加え、LUTのコンフィギュレーションのための機能実装も行い、トリガー回路を実機上で動かすための準備を整えた。また、統合の際に生じたタイミング違反を解決するため、パイプラインレジスタの実装方法の最適化やTrack Selectorのロジック最適化を行った。その結果、タイミング違反なくすべてのトリガーモジュールを統合することに成功した。

次に、SoCを活用したシングルボード試験システムを開発した。これにより、任意のデータセットに対するトリガー応答を、ハードウェア上で動作するトリガー回路を直接用いて検証することが可能となった。また、この試験システムとBitwiseシミュレーターを合わせることで、トリガー回路に対する詳細な調査およびデバッグが可能となり、トリガー開発における盤石な開発基盤を確立した。
本システムは開発段階だけでなく、本番運用時のファームウェアのメンテナンスやアップグレードに際した性能評価にも活用される予定で、精度の高いミューオントリガーの運転に必要不可欠なシステムとして機能する。

今後は、本試験で明らかになった無限運動量飛跡に対する3 \%程度のInefficiencyや、シングルミューオンMCに対する局所的な不具合に対する調査および修正を行い、TGC BW Coincidenceまでのトリガーロジックの精度を高めていく。また、トリガーロジックの拡張の観点では、Inner Coincidenceの開発を進めるとともに、本番運用に向けて堅牢性を高めていく。例えば、磁場内部検出器に不具合が生じた場合でも、トリガー効率を落とすことなく安定して運転できるよう、柔軟なトリガー運用オプションを準備する。

また本研究では、多くの要素技術を共有する形でPS board QAQC試験に向けた、コンパクトDAQシステムを開発した。まず、PS board上に搭載された素子とインターフェイスを網羅的に試験できるセットアップとして、JATHubを用いるのが合理的であることを見出し、試験セットアップを考案した。次に、ASDテストパルス試験とJTAG/Recovery/Monitor試験をベンチマークに、JATHubのPS領域とPL領域に必要な機能を実装し、テストベンチを用いたデモンストレーションを完了させた。最後にQAQC試験の高速化を目的とした、システム並列化を実現した。このシステムは、2024年から開始される1400枚以上に及ぶ量産試験で実際に使われる。さらに、その高い拡張性とコンパクトさから、TGC検出器の性能評価やエレクトロニクスの試験に幅広く応用されており、高輝度LHC-ATLAS実験におけるTGCシステムの安定運用を支える重要なインフラとなる。今後は2024年6月から始まるPS boardの量産に向けて、システムをさらに洗練させていく。

本研究で確立した、高速FPGAやSoCを活用した次世代型の検証モデルや、汎用的な読み出しシステムは、ATLAS実験などの高エネルギー物理実験のみに留まらず、様々なエレクトロニクスシステムに幅開く応用されることが期待される。


