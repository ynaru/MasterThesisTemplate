Large Hadron Collider (LHC) は、欧州原子核機構 (CERN) に建設された世界最高エネルギーの陽子陽子衝突型加速器である。LHCの衝突点の1つに設置されたATLAS検出器では、陽子陽子衝突で生成された粒子を観測することで、素粒子標準模型の精密測定や、標準模型を超えた新物理の探索を行っている。
2029年から始まる高輝度LHC-ATLAS実験では、高統計量を活かした物理探索を推し進めるため、最高瞬間ルミノシティが現行の約3倍の$7.5 \times 10^{34}\, \mathrm{cm^{-2}s^{-1}}$に増強される。加速器の高輝度化による衝突レートの増加に対応するため、ATLASのTrigger and Data Acquisition ( TDAQ ) システムもアップグレードされ、初段トリガーレートは1 MHzに、初段トリガーレイテンシーは10 $\mu$sに拡張される。
これに伴い、ATLAS検出器のエンドキャップ部分に位置するThin Gap Chamber (TGC) 検出器は、陽子バンチ衝突から飛来するミューオンを捉えるための読み出し・トリガーエレクトロニクスを一新する。主な変更点は検出器からのすべてのヒット信号をヒットの有無にかかわらず、前段回路から高速シリアル通信を用いて後段回路へ転送する点にあり、より柔軟かつ包括的なトリガー判定が可能となる。この実現のために、前段回路は新たにFPGAを搭載したものへ、後段回路は大規模FPGAとSystem on Chip (SoC)を搭載したものへ置き換えられる。

本研究では、後段回路上に実装されるトリガー論理回路の統合を完了し、その性能評価のための試験システムを開発した。
エンドキャップ部ミューオントリガーは後段回路上に大規模論理回路として実装され、6段階のトリガーモジュールをパイプライン的に接続することで実現される。各トリガーモジュール間の配線、トリガー動作に必要な周辺機能の実装、リソース使用量・タイミング制約を意識したロジック最適化を進め、トリガー回路を後段回路の全体ファームウェアへ統合した。
次に、統合したトリガー回路が実機上で正常に動作していること、また各トリガーモジュールが期待されるトリガー性能を有していることを検証するため、SoCを活用した次世代的なシングルボード試験システムを開発した。この試験システムでは、モンテカルロシミュレーションデータや実データを含む任意のデータセットを実機上のトリガー回路に投入し、それに対する各トリガーモジュールの出力を読み出すことができる。開発した試験システムを用いて、ミューオン飛跡のシミュレーションデータに対するトリガー効率を測定することで、実装されたロジックが期待されるトリガー性能を有しているか検証した。
本研究で開発した試験システムは、実際のハードウェア上で稼働するトリガー回路を本番運用に即した形で、直接試験できるという点で革新的である。このシステムを用いてトリガー回路を洗練していくことで、精度の高いミューオントリガーを実現する。さらに、このシステムは本番運用時のコミッショニングやトリガー回路のアップグレードにも活用され、高輝度LHC-ATLAS実験での効率的なデータ収集に貢献する。

また、多くの要素技術を共有する形で、前段回路に対する品質保証試験に向けて、SoCデバイスを活用したコンパクトな試験システムを開発した。はじめに、前段回路に搭載された素子とインターフェイスを網羅的に試験可能なセットアップとして、SoCデバイスを起点とした試験システムを設計した。このシステムではSoCデバイス1台を用いて1台の前段回路の試験を行い、SoC上に起動したLinuxを起点にすべての試験を完結させる。次に、試験のために必要なファームウェアとアプリケーションを開発し、テストベンチを用いたデモンストレーションにより、開発したシステムの動作検証を行った。最後に、品質保証試験の高速化を図るため、システムの並列化を達成し、2024年に開始される1500枚のボードの量産に必要なインフラシステムを完成させた。本研究で開発した試験システムは、その高い拡張性とコンパクトさから、TGC検出器の性能評価やエレクトロニクスの試験に幅広く応用されることが期待され、高輝度LHC-ATLAS実験においてTGCシステムの安定した運用を支える重要なインフラとなる。

本研究で確立した、高速FPGAやSoCを活用した次世代型の検証モデルや、汎用的な読み出しシステムは、将来の高エネルギー実験におけるにエレクトロニクスデザインに幅開く応用できると期待する。

